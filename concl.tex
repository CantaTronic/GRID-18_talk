\section{Conclusion}

\begin{frame}{Current status}
\begin{itemize}
%   \item The principal task for the multi-messenger astrophysics is to organize a joint analysis of data collected from different sources;
%   \item A joint Russian-German project of the KRAD is devoted to the investigation of this problem;
%   \item The objectives of the project include:
%   \begin{itemize}
%     \item a data center for astrophysical data;
%     \item software environment for the distributed analysis of this data;
%     \item open access to this data for everyone;
%     \item obtaining new knowledge about the laws of the universe by applying the most modern methods of machine learning to the big data collected;
%   \end{itemize}
%   \framebreak
  \item By now:
  \begin{itemize}
    \item The KASCADE project has a data center called KCDC, that is planned to serve as the basis for the future common center for data access;
    \item The differences in the data formats were analyzed and solutions for organizing storage and distributed data processing were proposed;
    \item A scheme of a relational database for the future data center is designed using a metadata-based approach;
  \end{itemize}
  \item The possibilities to apply the results of the project to educational and oureach activities are being explored; at present:
  \begin{itemize}
    \item KIT students use the KCDC for classes;
    \item The joint resource astroparticle.online is created to provide access to KASCADE and TAIGA data;
%     \item astroparticle.online is used by students of the Irkutsk State University in the course ``The fundamentals of modern astrophysics''
  \end{itemize}
\end{itemize}
\end{frame}

\begin{frame}{Conclusion}
\begin{itemize}
  \item Astroparticle physics: there are much data, but no unified storage system;
  \item KASCADE is the only astroparticle cosmic-ray experiment that has published all its data (on KCDC);
  \item Data structure is much different from LHC, a separate approach is needed;
  \item Two experiments (KASCADE and TAIGA): a new approach to data life cycle is being developed.
\end{itemize}
\end{frame}

\begin{frame}{}
  \begin{center}
    \textcolor{kit-green100}{\Huge Thank you\\for your attention!\vspace{1em}}

    \Large Any questions?
  \end{center}
\end{frame}
