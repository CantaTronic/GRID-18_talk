\section{Conclusion}

\begin{frame}{Current status}
% \begin{itemize}
%   \item The principal task for the multi-messenger astrophysics is to organize a joint analysis of data collected from different sources;
%   \item A joint Russian-German project of the KRAD is devoted to the investigation of this problem;
%   \item The objectives of the project include:
%   \begin{itemize}
%     \item a data center for astrophysical data;
%     \item software environment for the distributed analysis of this data;
%     \item open access to this data for everyone;
%     \item obtaining new knowledge about the laws of the universe by applying the most modern methods of machine learning to the big data collected;
%   \end{itemize}
%   \framebreak
%   \item By now:
  \begin{itemize}
    \item The KASCADE-Grande project has a data center called KCDC, that is planned to serve as the basis for the future common center for data access;
    \item The differences in the data formats were analyzed and solutions for organizing storage and distributed data processing were proposed;
    \item A scheme of a relational database for the future data center is designed using a metadata-based approach;
%   \end{itemize}
  \item The possibilities to apply the results of the project to educational and outreach activities are being explored.

    The joint resource \textbf{\textcolor{kit-green100}{astroparticle.online}} is created to provide access to KASCADE-Grande and TAIGA data and metadata.
%     \item astroparticle.online is used by students of the Irkutsk State University in the course ``The fundamentals of modern astrophysics''
  \end{itemize}
\end{frame}

\begin{frame}{Conclusion}
\small
\begin{itemize}
%   \item Astroparticle physics: there are much data, but no unified storage system;
  \item The constantly growing amount of accumulated astroparticle data and the request for the multi-messenger astronomy and machine learning, enable us to develop a unified system for astroparticle data storage and processing;
%   \item KASCADE is the only astroparticle cosmic-ray experiment that has published all its data (on KCDC);
  \item KASCADE-Grande is the only cosmic-ray experiment so far that has fully published its data and has a software infrastructure for data access and online analysis (KCDC);
  \item The pecularities of data format and acquisition make it impossible to utilize 'from scratch' the solutions widely used in collider experiments;
%   \item Two experiments (KASCADE and TAIGA): a new approach to data life cycle is being developed;
  \item We are developing a new approach to the astroparticle data life cycle for combined analysis of the KASCADE-Grande and TAIGA data;
  \item The built-up infrastructure will be used to analyze combined data sets with large statistics, allowing to study galactic sources of high-energy $\gamma$-rays, 
  which could be a notable step forward in multi-messenger astroparticle physics.

\end{itemize}
\end{frame}

\begin{frame}{}
  \begin{center}
    \textcolor{kit-green100}{\Huge Thank you\\for your attention!\vspace{1em}}

%     \Large Any questions?
  \end{center}
\end{frame}
