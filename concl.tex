\section{Conclusion}

\begin{frame}{Conclusion and outlook}
\small\vspace{-1.3ex}
\begin{itemize}
\setlength{\itemsep}{0pt}
\setlength{\parsep}{0pt}
\setlength{\parskip}{0pt}
  \item High-energy $\gamma$-rays investigations open up a fruitful opportunities for enriching knowledge about intragalactic cosmic ray sources and propagation and for a new physics search;
  \item We are developing stacked analysis methods (used, in particular, in recent works by Carpet-2) for investigation of KASCADE events, which could be associated with published HAWC sources with $E > 56$~TeV;
  \item There are two HAWC sources in the KASCADE field of view: \textit{2HWC\_2019+367} and \textit{2HWC\_J2013+415};
  \item For \textit{2HWC\_J2013+415} we calculated the expected gamma flux for energies of 56, 100~TeV and 1~PeV. The flux calculations for \textit{2HWC\_J2019+367} require accurate corrections on zenith angle and will be performed further;
  \item Employing the expected fluxes and the background isotropy estimation, we are going to study the possible background exceed within a 5$^\circ$ radius around the sources;
  \item Due to the energy scales compatibility of KASCADE/KASCADE-Grande and Tunka-133, the analysis below can be extended to data from the TAIGA experiment (Tunka valley, Russia).
\end{itemize}
\end{frame}

% \subsection{The end}
% \begin{frame}{}
%     \begin{center}
%         \textcolor{kit-green100}{\Huge Thank you\\for your attention!\vspace{1em}}  
%         \Large Any questions?
%     \end{center}
% \end{frame}
