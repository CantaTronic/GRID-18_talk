\section{Conclusion}

\begin{frame}{Current status}
% \begin{itemize}
%   \item The principal task for the multi-messenger astrophysics is to organize a joint analysis of data collected from different sources;
%   \item A joint Russian-German project of the KRAD is devoted to the investigation of this problem;
%   \item The objectives of the project include:
%   \begin{itemize}
%     \item a data center for astrophysical data;
%     \item software environment for the distributed analysis of this data;
%     \item open access to this data for everyone;
%     \item obtaining new knowledge about the laws of the universe by applying the most modern methods of machine learning to the big data collected;
%   \end{itemize}
%   \framebreak
%   \item By now:
  \begin{itemize}
    \item The KASCADE project has a data center called KCDC, that is planned to serve as the basis for the future common center for data access;
    \item The differences in the data formats were analyzed and solutions for organizing storage and distributed data processing were proposed;
    \item A scheme of a relational database for the future data center is designed using a metadata-based approach;
%   \end{itemize}
  \item The possibilities to apply the results of the project to educational and oureach activities are being explored.
%   ; at present:
%   \begin{itemize}
%     \item 
%     KIT students use the KCDC for classes;
%     \item 

    The joint resource \textbf{\textcolor{kit-green100}{astroparticle.online}} is created to provide access to KASCADE and TAIGA data.
%     \item astroparticle.online is used by students of the Irkutsk State University in the course ``The fundamentals of modern astrophysics''
  \end{itemize}
% \end{itemize}
\end{frame}

\begin{frame}{Conclusion}
\small
\begin{itemize}
%   \item Astroparticle physics: there are much data, but no unified storage system;
  \item The constantly growing amount of data accumulated by modern detectors in the field of 
  astroparticle physics, as well as the request for the multi-messenger astronomy and the use of 
  machine learning methods in this field, makes it urgent to create a unified system for storing 
  and processing astroparticle-physics data;
%   \item KASCADE is the only astroparticle cosmic-ray experiment that has published all its data (on KCDC);
  \item At the moment, KASCADE-Grande is the only experiment in the field of astroparticle physics that 
  has fully published its data and has a software infrastructure for data access and online analysis.
  \item The specificity of data acquisition of the astroparticle experiment and the format of these data 
  makes it impossible to utilize from scratch the solutions developed in collider experiments.
%   \item Two experiments (KASCADE and TAIGA): a new approach to data life cycle is being developed;
  \item We are developing a new approach to the astroparticle data life cycle for combined analysis of the KASCADE and TAIGA data.
  \item The built-up infrastructure will be used for a search with large statistics and an improved method for 
  high-energy $\gamma$-rays is going to perform an important step in multi-messenger astroparticle physics of galactic sources of cosmic rays.  

\end{itemize}
\end{frame}

% \begin{frame}{}
%   \begin{center}
%     \textcolor{kit-green100}{\Huge Thank you\\for your attention!\vspace{1em}}
%
%     \Large Any questions?
%   \end{center}
% \end{frame}
