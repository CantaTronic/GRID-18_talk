% Application server: 
%         TODO: conception of the aggregation server!!! 
%         computing sources (TODO: which sources can we use to run our application in parallel?)

\begin{frame}{Application server concept in GRADLC}
\textbf{Aim}: to provide user the opportunity to analyze the data selected remotely.

\textbf{Computing sources}: CR local network, clusters (GRIDKa, BW-HPC, etc.), external clouds (Exoscale, OpenNebula, Amazon, Google, etc.).

\textbf{Possible solutions}: 
  \begin{itemize}
    \item HTCondor or HTCondor-based workload management systems: VCondor, Panda, Dirac;
    \item Data mapping plugins;
    \item Metadata DB for faster search;
    \item Data caching on aggregation server
  \end{itemize}

\end{frame}

%  ANALYSIS 
% \begin{frame}{Analysis use-cases}
%     \includegraphics[width=1\textwidth]{pics/an_steps.pdf}
%     \vspace{2em}
%     \begin{itemize}
%         \item Analysis could be either algorithmic or machine learning;
%         \item Machine learning requires large enough statistics\\in order to work properly.
%     \end{itemize}
% \end{frame}

% \begin{frame}{Analysis features}
% \vspace{-3em}
% \begin{block}{Software for data analysis depends on a particular experiment}
%   \begin{itemize}
%     \item Problem: It may even require dedicated system environment
%     \item \textbf{Solution: Virtualization\footnotemark[2]}
%   \end{itemize}
% \end{block}

% \begin{block}{Data analysis requires huge amounts of input data}
%   \begin{itemize}
%     \item Problem: It is often more optimal to perform it on the same site the data are stored
%     \item \textbf{Solution: Job management}
%   \end{itemize}
% \end{block}
% \insimg{ADLC1.pdf}
%   \footnotesize\footnotetext[2]{``The act of creating a virtual (rather than actual)
%   version of something,\\ \hspace{1.5em}including virtual computer hardware platforms,
%   storage devices,\\ \hspace{1.5em}and computer network resources''. $\copyright$ Wikipedia}
% \end{frame}

% %TODO: POSSIBLE SOLUTIONS
% \begin{frame}{Possible solutions \\for the application level}
%  
% \end{frame}
