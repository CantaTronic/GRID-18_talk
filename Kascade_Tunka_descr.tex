%experiments overview

\begin{frame}{KASCADE-Grande}
\begin{itemize}
  \item Proposed in 1989---disassembled in 2013;
  \item Aimed at studying
  high-evergy (galactic) cosmic rays by observing extensive air showers (EAS);
%   processes at the edge of the Galaxy and beyond by observing extended atmospheric showers (EAS);
  \item Consisted of:
  \begin{itemize}
    \item scintillator arrays:
%     detecting $e$, $\gamma$, $\mu$:
    \begin{itemize}
  %сцинтиляторы, различают e, gamma, mu
    \item KASCADE---256 stations;
    \item GRANDE---37 stations;
    \end{itemize}
 %один большой калориметр
    \item Hadronic callorimeter;
 %радиодетектор
    \item Digital radio array LOPES;
%     detecting $e$, $e^{+}$;
% позволяющих наблюдать различные компоненты ливня
  \end{itemize}
  \item Important features of cosmic-ray spectrum have been obtained. The data analysis is ongoing;
%  благодаря данным с эксперимента было открыто много всего ополезного, при этом анлиз данных продолжается. новые статьи выходят
  \item KCDC (\textbf{K}ASCADE \textbf{C}osmic Ray \textbf{D}ata \textbf{C}enter, \textcolor{blue}{\texttt{http://kcdc.ikp.kit.edu}}) is a dedicated portal where all the data collected are available online. % At the moment
\end{itemize}

\begin{tikzpicture}[remember picture,overlay]
  \node[xshift=-12ex,yshift=-21ex] at (current page.north east){%
    \includegraphics[width=0.3\textwidth]{pics/KCDC-Logo.png}
  };
\end{tikzpicture}
% \parbox[t][0pt]{0pt}{
%   \vspace{-0.63\textheight}
%   ~\hspace{0.68\textwidth}\includegraphics[width=0.3\textwidth]{pics/KCDC-Logo.png}
% }
\end{frame}

\begin{frame}{TAIGA - Tunka Advanced Instrument for cosmic ray physics and Gamma Astronomy}
% \footnotesize
% % \vspace{-1em}
\begin{itemize}
 \item The detectors construction started in 90s with Tunka-25 setup;
 \item Changed name from Tunka to TAIGA;
 \item Is ongoing and continiously enhancend;
%  \item Currently consists of 4 detectors presented + TUNKA IACT is under construction;
\end{itemize}

% \vspace{2em}

\begin{center}
    \includegraphics[width=1\textwidth]{pics/TAIGA_exp_wt.pdf} 
\end{center}

\end{frame}

